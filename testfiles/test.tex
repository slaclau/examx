\documentclass[
    style=OCRALevel,
%    coverstyle=WellyOCR,
%    answerbookstyle=OCRALevel,
%    solutionstyle=space
    ]{examx}
\examtitle{Test of examx}
\examtopic{topic}
\examcode{code}
\examdate{date}
\examtime{time}

%\printanswers

\begin{document}
\examcover
\begin{questionpaper}
\begin{questions}
    \question[5][4] This is a sample question with no parts.
    
    \solnor{TEST}
    
    \question This is a sample question.
    \begin{parts}
        \part[3][5] This is a sample part.
        
        \part[1][3] This is a sample part.
    \end{parts}
    
    \solnor{
        \begin{parts}
            \part This is a sample solution.
            \part So is this.
        \end{parts}
    }
    
    \question[2] This is another sample question
    
    \begin{parts}
        \part[4][15] This is another sample part.
        
        \part[3][15] So is this.
    \end{parts}

    \examnewpage
    This is a new page unless the option shortpaper is used or this is 
    the markscheme.
    \question This is a sample question.
    
    \begin{parts}
        \part[3][6] This is a sample part.
        \part[4][8] So is this.
        \part[2][6] And this.
    \end{parts}
    
    
    \msexamnewpage
    This is a new page unless the option shortpaper is used.
    
    \question This is a sample question with parts and subparts.
        
    \begin{parts}
        \part[-1][8] This is a part with subparts (and no points of 
its 
        own).
        \begin{subparts}
            \subpart[5] This is a subpart with points.
            \subpart[4] This is another subpart.
        \end{subparts}
        
        \part[3][6] This is a sample part.
        \part[1][2] And again.
        \part[1][2] And again.
    \end{parts}
\end{questions}
\end{questionpaper}
\answerbook
\end{document}