% \iffalse meta-comment
%
% Copyright (C) 2022 by Sebastien Laclau
% -----------------------------------
%
% This file may be distributed and/or modified under the
% conditions of the LaTeX Project Public License, either version 1.3c
% of this license or (at your option) any later version.
% The latest version of this license is in:
%
% http://www.latex-project.org/lppl.txt
%
% and version 1.3c or later is part of all distributions of LaTeX
% version 2008 or later.
%
% \fi
%
% \iffalse

% \section{Identification}
%
%    Announce the file name and its version:
%
%    \begin{macrocode}
%<*driver>
\ProvidesFile{examx-userguide.dtx}[2022/02/01 v1.0.1
%    \end{macrocode}
%
%\section{Driver}
%
% The next bit of code contains the documentation driver file for
% \TeX{}, i.e., the file that will produce the documentation you are
% currently reading. It will be extracted from this file by the
% {\sc docstrip} program.
%
%    \begin{macrocode}
%<*driver>
]
\documentclass{ltxdoc}
%    \end{macrocode}
%    Some things do not need indexing.
%    \begin{macrocode}
%    \end{macrocode}
%    We do want an index, using line numbers, and a change log.
%    \begin{macrocode}
\EnableCrossrefs
\CodelineIndex
\RecordChanges
%    \end{macrocode}
%    The following code retrieves the date and version information from
%the file.
%    \begin{macrocode}
\GetFileInfo{examx-userguide.dtx}
%    \end{macrocode}
%    Here are some commonly used abbreviations:
%    \begin{macrocode}
% SVN keywords: $Revision$

\providecommand*{\Lopt}[1]{\textsf {#1}}         % typeset an option
\providecommand*{\file}[1]{\texttt {#1}}         % typeset a file
\providecommand*{\Lcount}[1]{\textsl {\small#1}} % typeset a counter
\providecommand*{\pstyle}[1]{\textsl {#1}}       % typeset a pagestyle
\providecommand*{\Lenv}[1]{\texttt {#1}}         % typeset an 
%environment
\providecommand*{\Lpack}[1]{\textsf {#1}}        % typeset a package
%    \end{macrocode}
%    We also want the full details.
%    \begin{macrocode}
\begin{document}
	\DocInput{examx-userguide.dtx}
    \PrintIndex
    \PrintChanges
\end{document}
%</driver>
%    \end{macrocode}
%
% \fi
%
% \CheckSum{0}
%
% \changes{v1.0}{2022/02/01}{Initial version}
% \changes{v1.0.1}{2022/02/03}{Start of semver}
%
%
%
% \title{User guide for the \textsf{examx} class\thanks{This document
    % corresponds to \file{examx-userguide.dtx}~\fileversion,
    % dated \filedate.}}
% \author{Sebastien Laclau \\ \texttt{slaclau@wellingtoncollege.org.uk}
%\\ \texttt{seb.laclau@gmail.com}}
%
% \maketitle
% \tableofcontents
%
% \section{Basic usage}
% \subsection{Document structure}
% A basic document using the \Lpack{examx} class looks like the following:
%    \begin{macrocode}
\documentclass{examx}

\begin{document}
    \examcover
    \begin{questionpaper}

    \end{questionpaper}
    \answerbook
\end{document}
%    \end{macrocode}
% All \Lenv{questions} environments (which are from the \Lpack{exam}
%class) must be enclosed in the \Lenv{questionpaper} environment. This
%environment ensures the points are added together for each question and
%formats the beginning and end of the paper.
%
% The |\examcover| commands inserts a cover and similarly the
%|\answerbook| command inserts an answerbook.
%
% \subsection{Questions}
%
% Within this environment, the \Lenv{questions} environment may be used
%multiple times. It contains |\question| commands which may be used with
%between zero and three.
%
% The first argument is the number of points for the question, the second
%is the number of lines, and the third the amount of blank space. These
%values apply only to answer space produced by |\answerbook|.
%
% The \Lenv{parts} and \Lenv{subparts} environments work as in the
%\Lpack{exam} class except that the additional arguments taken by
%|\question| may also be used for |\part| and |\subpart|.
%
% \subsection{Solution commands}
% The following commands may be used to insert various types of solution
%space:
% \begin{itemize}
%    \item |\solnor| -- this produces an answer space depending on class
%options.
%    \item |\solnordots| -- this produces dotted lines.
%    \item |\solnorlines| -- this produces lines.
%    \item |\solnorgrid| -- this produces a grid.
%    \item |\solnorspace| -- this produces a blank space.
% \end{itemize}
%
% In general, only the first of these should be used.
%
% If a whole page of answer space is needed, then the |\solnpage| command
%should be used. It works in the same way as |\solnor|.
%
% \section{Exam metadata}
%
% The following commands may be used to configure metadata -- this
%metadata is usually included on covers for both the question paper and
%the answer booklet but may be used as needed by the styles.
% \begin{itemize}
%    \item |\institution{arg}| -- the institution setting the exam.
%    \item |\department{arg}| -- the department within the institution.
%    \item |\logo{arg}| -- the logo of the institution or department --
%this will usually be the name of a grahics file located somewhere
%searchable by \TeX.
% \end{itemize}
%
% \section{Configuration}
% \subsection{Class options}
%
% \section{Styles}
%
% \StopEventually{}
%
% \Finale