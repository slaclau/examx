% \iffalse meta-comment
%
% Copyright (C) 2022 by Sebastien Laclau
% -----------------------------------
%
% This file may be distributed and/or modified under the
% conditions of the LaTeX Project Public License, either version 1.3
% of this license or (at your option) any later version.
% The latest version of this license is in:
%
% http://www.latex-project.org/lppl.txt
%
% and version 1.3 or later is part of all distributions of LaTeX
% version 2005/12/01 or later.
%
% \fi
%
% \iffalse

% \section{Identification}
%
%    Announce the file name and its version:
%
%    \begin{macrocode}
%<default&all>\ProvidesFile{default.clo}
%<OCRALevel&all>\ProvidesFile{OCRALevel.clo}
%<OCRALevel&answerbook>\ProvidesFile{OCRALevelanswerbook.clo}
%<advanced&answerbook>\ProvidesFile{advancedanswerbook.clo}
%<veryadvanced&answerbook>\ProvidesFile{veryadvancedanswerbook.clo}
%<WellyOCR&covers>\ProvidesFile{WellyOCRcovers.clo}
%<*driver>
\ProvidesFile{styles.dtx}
%</driver>
	[2022/01/31 v1.0
%<!driver> standard latex class option file]
%<*driver>
]
\documentclass{ltxdoc}
%    \end{macrocode}
%    Some things do not need indexing.
%    \begin{macrocode}
\DoNotIndex{\AtEndOfClass, \AtEndEnvironment}
\DoNotIndex{\def, \gdef, \xdef, \let, \relax, \newif, \newcommand,  
\renewcommand, \NewDocumentCommand, \RenewDocumentCommand}
\DoNotIndex{\,, \\, \vspace, \par, \baselineskip}
\DoNotIndex{\hline}
\DoNotIndex{\normalsize, \large, \Large, \LARGE, \huge}
\DoNotIndex{\bfseries, \textsc, \textcolor, \rm, \bf}
\DoNotIndex{\csname{else}, \cs{fi}}
\DoNotIndex{\csname{loop}, \csname{repeat}, \foreach, \breakforeach, 
\m, \n, \x, \s}
\DoNotIndex{\begin, \end}
\DoNotIndex{\arabic, \alph, \Alph}
\DoNotIndex{\newcounter, \setcounter, \pgfmathsetcounter, \addtocounter}
\DoNotIndex{\lpos, \mpos, \rpos, \blankrow, \pageendrule, 
\pagestartrule}
\DoNotIndex{\csname, \endcsname}
%    \end{macrocode}
%    We do want an index, using line numbers, and a change log.
%    \begin{macrocode}
\EnableCrossrefs
\CodelineIndex
\RecordChanges
%    \end{macrocode}
%    The following code retrieves the date and version information from 
%the file.
%    \begin{macrocode}
\GetFileInfo{styles.dtx}
%    \end{macrocode}
%    Here are some commonly used abbreviations:
%    \begin{macrocode}
\providecommand{\Lopt}[1]{\textsf {#1}}
\providecommand{\file}[1]{\texttt {#1}}
\providecommand{\Lcount}[1]{\textsl {\small#1}}
\providecommand{\pstyle}[1]{\textsl {#1}}
%    \end{macrocode}
%    We also want the full details.
%    \begin{macrocode}
\begin{document}
	\DocInput{styles.dtx}
	\PrintIndex
	\PrintChanges
\end{document}
%</driver>
%    \end{macrocode}
%
% \fi
%
% \CheckSum{0}
%
% \changes{v1.0}{2022/01/31}{Initial version}
%
%
%
%
% \title{Styles for the \textsf{examx} class\thanks{This document
	% corresponds to \file{styles.dtx}~\fileversion,
	% dated \filedate.}}
% \author{Sebastien Laclau \\ \texttt{slaclau@wellingtoncollege.org.uk} \\ \texttt{seb.laclau@gmail.com}}
%
% \maketitle
%
% \StopEventually{\PrintIndex \PrintChanges}
% \section{The {\sc docstrip} modules}
%
% The following modules are used in the implementation to direct
% {\sc docstrip} in generating the external files:
% \begin{center}
    % \begin{tabular}{ll}
        %   class  & produce the documentclass \textsf{examx}\\
        %   driver & produce a documentation driver file \\
        % \end{tabular}
    % \end{center}
%
%\section{Complete styles}
%\subsection{Default style}
%<*default>
%    \begin{macrocode}
\solndots
%    \end{macrocode}
%</default>
%\subsubsection{Answer book}
%    \begin{macrocode}
%<*default&(answerbookbackcover|answerbook|all)>
\renewcommand{\answerbookbackcover}{
    \begin{center}
        \large\bf ADDITIONAL ANSWER SPACE
    \end{center}
    If additional space is required, you should use the following lined 
    page(s). The question number(s) must be clearly shown in the 
    margin(s).
    \fillwithdottedlines{\stretch{1}}
}
\renewcommand{\doubleanswerbookbackcover}{
    \begin{center}
        \large\bf ADDITIONAL ANSWER SPACE
    \end{center}
    If additional space is required, you should use the following lined 
    page(s). The question number(s) must be clearly shown in the 
    margin(s).
    \fillwithdottedlines{\stretch{1}}
    \newpage
    \fillwithdottedlines{\stretch{1}}
}
%</default&(answerbookbackcover|answerbook|all)>
%    \end{macrocode}
%\subsection{OCRALevel style}
%\subsubsection{Initial code}
%    The initial code configures various features of the examx 
%class.
%    \begin{macrocode}
%<*OCRALevel>
\@alwaysdroppointsatright
\soln

\sisetup{inter-unit-product = \,} %Configures siunitx to use a 
%dot between units

\setmainfont{Times New Roman} 

\titleformat{\section}{\centering\bf}{Section \Alph{section}: 
}{0pt}{}

\geometry{left=0.86 in,right=1.14 in,top=1 in,bottom=1 in}
\savegeometry{main}
\newgeometry{left=0.68 in,right=0.68 in,top=1 in,bottom=1 in}
\savegeometry{answerbook}
\loadgeometry{main}


\footer{}{}{\bf\oddeven{Turn over}{}}
\header{}{\bf \thepage}{}

\renewcommand{\questionpaperend}{
    \footer{}{}{}
    \begin{center}
        \ifprintanswers {\bf END OF MARKSCHEME} \else {\bf END OF 
        QUESTION PAPER} \fi
        \oddeven{\newpage {\bf This page is intentionally 
        blank}}{}
    \end{center}
}
\renewcommand{\questionpaperstart}{
    \ifprintanswers \else
    \begin{center}
        Answer {\bf all} the questions.
    \end{center}
    \fi
}
\renewcommand{\questionshook}{\setlength{\itemsep}{1in}}
\renewcommand{\partshook}{\setlength{\topsep}{\baselineskip} 
\setlength{\itemsep}{\baselineskip}}
\renewcommand{\solnorhook}{
    \renewcommand{\partshook}{\setlength{\itemsep}{\baselineskip}}
}

\renewcommand{\questionlabel}{\bfseries\thequestion.}
\renewcommand{\partlabel}{\bfseries(\thepartno)}
\renewcommand{\subpartlabel}{\bfseries(\thesubpart)}
\renewcommand{\subsubpartlabel}{\bfseries(\thesubsubpart)}

\pointname{}
\pointformat{\bfseries[\thepoints]}
%</OCRALevel>
%    \end{macrocode}

%\subsubsection{Answer book}

%    The remaining code concerns the answer book.
%    \begin{macrocode}
%<*OCRALevel&(answerbook|all)>
\newcounter{numberofquestions}
\newcounter{maxlinesinpage}
\setcounter{maxlinesinpage}{31}
\newcounter{currentline}
\newcounter{currentlineinquestion}
\newcounter{questionnumber}
\newcounter{numlinesinquestion}
\newcounter{partlevel}

\newcounter{tempa}
\newcounter{tempb}
\newcounter{tempc}
\newcounter{linesleft}
\newcounter{lastques}

\newif\ifquestioncomplete
\def\lpos{0}
\def\mpos{0.59 in}
\def\rpos{6.89 in}
\def\lineheight{.76}

\newcommand{\blankrow}{
    \draw (\mpos,-1-\thecurrentline)--(\rpos,-1-\thecurrentline);
    \draw (\lpos,-\thecurrentline)--(\lpos,-1-\thecurrentline);
    \draw (\mpos,-\thecurrentline)--(\mpos,-1-\thecurrentline);
    \draw (\rpos,-\thecurrentline)--(\rpos,-1-\thecurrentline);
}
\newcommand{\pagestartrule}{
    \draw (\lpos,0)--(\rpos,0);
}
\newcommand{\pageendrule}[1][\themaxlinesinpage]{
    \draw (\lpos,-#1)--(\rpos,-#1);
}

\renewcommand{\answerbookpages}{
    \newcounter{int}
    
    \newcounter{arraycard}
    \def\arrayLength##1{%
        \setcounter{arraycard}{0}%
        \foreach \x in ##1{%
            \stepcounter{arraycard}%
        }%
        %
    }
    \newcommand{\setcountertoarraylength}[2]{
        \arrayLength{##2}
        \setcounter{##1}{\thearraycard}
    }
        
    \footer{}{}{}
    \loadgeometry{answerbook}
    \setcountertoarraylength{numberofquestions}{\@questionorpartnumber}
    \addtocounter{numberofquestions}{-1}
    
    \xdef\qs{{\@questionorpartnumber}}
    \xdef\lines{{\@lines}}
    \xdef\ip{{\@partlevel}}
    
    \setcounter{numlinesinquestion}{0}
    \setcounter{int}{1}
    \setcounter{tempa}{1}
    \setcounter{linesleft}{0}
    \setcounter{currentlineinquestion}{1}	
    
    \loop
    \begin{center}
        \begin{tikzpicture}[yscale=\lineheight]
        \pagestartrule
        \setcounter{currentline}{0}	
        \foreach \n in {1,...,\themaxlinesinpage}{
            \blankrow
            \addtocounter{currentline}{1}
        }
        \pageendrule
        
        \setcounter{currentline}{0}
        \setcounter{int}{\thetempa}
        \foreach \n in {\theint,...,\thenumberofquestions}{
            \pgfmathsetcounter{questionnumber}{\qs[\n]}
            \pgfmathsetcounter{partlevel}{\ip[\n]}
            \ifnum \thelinesleft>0
            \draw[draw=none] 
            (\mpos,-1)--(\rpos,-1)node[pos=0.08,above]{\bf 
            (continued)};
            \setcounter{numlinesinquestion}{\thelinesleft}
            \else
            \ifnum \thecurrentline=0 \else \ifnum \thepartlevel>0 
            \draw[very thick] 
            (\lpos,-\thecurrentline)--(\rpos,-\thecurrentline); 
            \fi \fi
            \pgfmathsetcounter{numlinesinquestion}{\lines[\n]}
            \fi
            
            \ifnum \thecurrentline=\themaxlinesinpage \else
            \ifnum \thepartlevel=-1
            \draw 
            (\lpos,-1-\thecurrentline)--(\rpos,-1-\thecurrentline)node[pos=0.5,above]{\bf
             Section \Alph{questionnumber}: 
            \nameref{sec:\Alph{questionnumber}}};
            \ifnum \thecurrentline=0 \draw[white,thick] 
            (\lpos,0)--(\rpos,0); \fi
            \draw[white,thick] 
            (\lpos,-\thecurrentline)--(\lpos,-1-\thecurrentline);
            \draw[white,thick] 
            (\mpos,-\thecurrentline)--(\mpos,-1-\thecurrentline);
            \draw[white,thick] 
            (\rpos,-\thecurrentline)--(\rpos,-1-\thecurrentline);
            \fi \ifnum \thepartlevel=0
            \setcounter{lastques}{\thequestionnumber}
            \ifnum \thenumlinesinquestion=0 \else 
            \draw[draw=none] 
            (\lpos,-1-\thecurrentline)--(\mpos,-1-\thecurrentline)node[pos=0.65,above]{\bf
             \thelastques}; \fi
            \fi \ifnum \thepartlevel=1
            \draw[draw=none] 
            (\lpos,-1-\thecurrentline)--(\mpos,-1-\thecurrentline)node[pos=0.65,above]{\bf
             \thelastques(\alph{questionnumber})};
            \fi
            \fi
            \addtocounter{tempa}{1}
            \addtocounter{currentline}{\thenumlinesinquestion}
            \setcounter{linesleft}{0}
            
            \ifnum \n=\thenumberofquestions \else
            \setcounter{tempb}{\n}
            \addtocounter{tempb}{1}
            \pgfmathsetcounter{tempb}{\ip[\thetempb]}
            
            \ifnum \thetempb=0\ifnum \thepartlevel=-1 \else 
            \breakforeach \fi \fi
            \fi
            \ifnum \thecurrentline>\themaxlinesinpage
            \addtocounter{tempa}{-1}
            \setcounter{linesleft}{\thecurrentline}
            \addtocounter{linesleft}{-\themaxlinesinpage}
            \draw[draw=none] 
            (\mpos,-\themaxlinesinpage)--(\rpos,-\themaxlinesinpage)node[pos=0.77,above]{\bf
             (answer space continued on next page)};
            \breakforeach
            \fi
        }
        \ifnum \thelinesleft=0
        \draw[very thick] 
        (\lpos,-\themaxlinesinpage)--(\rpos,-\themaxlinesinpage);
        \fi
        \oddeven{\draw[draw=none] 
        (\mpos,-1-\themaxlinesinpage)--(\rpos,-1-\themaxlinesinpage)node[pos=0.9,above]{\bf
         Turn over};}{}
        \end{tikzpicture}
    \end{center}
    \newpage
    \ifnum \thetempa<\thenumberofquestions
    \repeat
    \loadgeometry{main}
}

\renewcommand{\answerbookbackcover}{
    \loadgeometry{answerbook}
    \begin{center}
        \large\bf ADDITIONAL ANSWER SPACE
    \end{center}
    If additional space is required, you should use the following 
    lined page(s). The question number(s) must be clearly shown 
    in the margin(s).
    
    \begin{center}
        \begin{tikzpicture}[yscale=\lineheight]
        \pagestartrule
        \setcounter{currentline}{0}	
        \foreach \n in {1,...,21}{
            \blankrow
            \addtocounter{currentline}{1}
        }
        \pageendrule[21]
        \end{tikzpicture}
    \end{center}
    \loadgeometry{main}
}

\renewcommand{\doubleanswerbookbackcover}{
    \loadgeometry{answerbook}
    \begin{center}
        \large\bf ADDITIONAL ANSWER SPACE
    \end{center}
    If additional space is required, you should use the following 
    lined page(s). The question number(s) must be clearly shown 
    in the margin(s).
    
    \begin{center}
        \begin{tikzpicture}[yscale=\lineheight]
        \pagestartrule
        \setcounter{currentline}{0}	
        \foreach \n in {1,...,25}{
            \blankrow
            \addtocounter{currentline}{1}
        }
        \pageendrule[25]
        \end{tikzpicture}
    \end{center}
    \newpage
    \begin{center}
        \begin{tikzpicture}[yscale=\lineheight]
        \pagestartrule
        \setcounter{currentline}{0}	
        \foreach \n in {1,...,28}{
            \blankrow
            \addtocounter{currentline}{1}
        }
        \pageendrule[28]
        \end{tikzpicture}
    \end{center}
    \loadgeometry{main}
}
%</OCRALevel&(answerbook|all)>
%    \end{macrocode}

%\section{Incomplete styles}

%\subsection{Cover styles}

%\subsubsection{Welly OCR style}
%    This is the initial code.
%    \begin{macrocode}
%<*WellyOCR&covers>
\newgeometry{margin=1 in}
\savegeometry{cover}
\loadgeometry{main}
\input{OCRALevelformulaesheets.def}
%    \end{macrocode}
%\begin{macro}{\examcover}
%   We renew the |\examcover| macro in a style similar to an OCR 
%paper.
%    \begin{macrocode}
\renewcommand{\examcover}{
    {\setmainfont{\coverfont}
        \loadgeometry{cover}
        \thispagestyle{empty}
        \begin{center}
            \includegraphics[scale=0.25]{Welly} \par
            \vspace{0.25in}
            {\bf
                \huge\textsc{{Wellington College}} \par
                \huge\textsc{{Mathematics Department}} \par}
        \end{center}
        \begin{flushleft}
            \bf
            \vspace{0.25in}
            \LARGE \@examtitle \par
            \Large \@examtopic \par
            \LARGE \@examdate \par
            \ifprintanswers \Large Marks available: 
            \pointsinrange{Core} \par \else \Large Time allowed: 
            \@examtime \par \fi
            \normalsize
            \vspace{0.25in}
            \ifprintanswers
            {\huge\textcolor{red}{MARKSCHEME}}
            \else
            \vspace{0.01in}
            \rm
            \normalsize
            {\bf INSTRUCTIONS}
            \begin{itemize}
                \itemsep-1pt
                \item Use black ink. HB pencil may be used for 
                graphs and diagrams only.
                \item Complete the boxes provided on the Printed 
                Answer Booklet with your name, and your teacher's 
                name.
                \item Answer {\bf all} the questions.
                \item {\bf Write your answer to each question in 
                the space provided in the Printed Answer 
                Booklet.} If additional space is required, you 
                should use the lined page(s) at the end of the 
                Printed Answer Booklet. The question number(s) 
                must be clearly shown.
                \item \ifcalculator You are permitted to use a 
                scientific or graphical calculator in this paper. 
                \else You are not permitted to use a calculator 
                in this paper. \fi
                \item Give non-exact numerical answers correct to 
                3 significant figures unless a different degree 
                of accuracy is specified in the question.
                \item The acceleration due to gravity is denoted 
                by $\grav\,\unit{\m\per\squared\s}$. Unless 
                otherwise instructed, when a numerical value is 
                needed, use $\grav=9.8$.
            \end{itemize}
            
            {\bf INFORMATION}
            \begin{itemize}
                \itemsep-1pt
                \item The total number of marks for this paper is 
                {\bf \pointsinrange{Core}}.
                \item The marks for each question are shown in 
                brackets {\bf [ ]}.
                \item {\bf You are reminded of the need for clear 
                presentation in your answers.}
                \item \ifanswerbook The Printed Answer Booklet 
                consists of {\bf \pageref{LastPage}} pages.\fi 
                The Question Paper consists of {\bf 
                \pageref{lastpageofquestionpaper}} pages.
            \end{itemize}
            \fi
        \end{flushleft}
        \vspace{0.25in}
        \examnewpage
        \loadgeometry{main}}
    \ifprintanswers \else \ifexamx@formulae
    \formulaesheet \fi \fi
}
%    \end{macrocode}
%\end{macro}
%\begin{macro}{\answerbookcover}
%   We renew the |\answerbookcover| macro in a style similar to 
%an OCR paper.
%    \begin{macrocode}
\renewcommand{\answerbookcover}{
    {\setmainfont{\coverfont}
        \loadgeometry{cover}
        \thispagestyle{empty}
        \begin{center}
            \includegraphics[scale=0.25]{Welly} \par
            \vspace{0.25in}
            {\bf
                \huge\textsc{{Wellington College}} \par
                \huge\textsc{{Mathematics Department}} \par}
        \end{center}
        \begin{flushleft}
            \bf
            \vspace{0.25in}
            \LARGE \@examdate \par
            \Large \@examtitle \par
            \large \@examtopic \par
            \normalsize
            PRINTED ANSWER BOOKLET \par
            Time allowed: \@examtime \par
            \vspace{0.25in}
            {\renewcommand{\arraystretch}{2}
                \begin{tabular}{|m{2.5cm}|m{10cm}|}
                    \hline
                    Name: &\\
                    \hline
                    Teacher: &\\
                    \hline
            \end{tabular}}
            
            \vspace{0.25in}
            \rm
            \normalsize
            {\bf INSTRUCTIONS}
            \begin{itemize}
                \itemsep-1pt
                \item The Question Paper will be found inside the 
                Printed Answer Booklet.
                \item Use black ink. HB pencil may be used for 
                graphs and diagrams only.
                \item Answer {\bf all} the questions.
                \item {\bf Write your answer to each question in 
                the space provided in the Printed Answer 
                Booklet.} If additional space is required, you 
                should use the lined page(s) at the end of 
                the Printed Answer Booklet. The question 
                number(s) must be clearly shown.
                \item \ifcalculator You are permitted to use a 
                scientific or graphical calculator in this paper. 
                \else You are not permitted to use a calculator 
                in this paper. \fi
                \item Give non-exact numerical answers correct to 
                3 significant figures unless a different degree 
                of accuracy is specified in the question.
                \item The acceleration due to gravity is denoted 
                by $\grav\,\unit{\m\per\squared\s}$. Unless 
                otherwise instructed, when a numerical value is 
                needed, use $\grav=9.8$.
            \end{itemize}
            
            {\bf INFORMATION}
            \begin{itemize}
                \itemsep-1pt
                \item {\bf You are reminded of the need for clear 
                presentation in your answers.}
                \item The Printed Answer Booklet consists of {\bf 
                \pageref{LastPage}} pages. The Question Paper 
                consists of 
                {\bf \pageref{lastpageofquestionpaper}} pages.
            \end{itemize}
        \end{flushleft}
        \vspace{0.25in}
        \loadgeometry{main}}
}
%</WellyOCR&covers>
%    \end{macrocode}
%\end{macro}

%\subsection{Answer book styles}

%\subsubsection{Advanced answer book}
%    \begin{macrocode}
%<*advanced&answerbook>
\newcounter{numpofq}
\newcommand{\setcountertopartsofq}[1]{%
    \def\qnotemp{#1}%
    \setcounter{numpofq}{1}%
    % We go to \numpofqrelay to increment numpofq until we find
    % a part number that doesn't exist:
    \numpofqrelay
}% setcountertopartsofq

\def\numpofqrelay{%
    \expandafter\ifx\csname Pg@part@\qnotemp
    @\arabic{numpofq}\endcsname\relax
    % This part number doesn't exist; back up one and exit:
    \addtocounter{numpofq}{-1}%
    \let\nextnumpofqrelay=\relax
    \else
    % This part number exists; try the next number:
    \addtocounter{numpofq}{1}%
    \let\nextnumpofqrelay=\numpofqrelay
    \fi
    \nextnumpofqrelay
}% numpofqrelay

\renewcommand{\answerbookpages}{
    \newcounter{int}
    \newcounter{iint}
    \setcounter{int}{0}
    \loop
    \addtocounter{int}{1}
    \setcountertopartsofq{\theint}
    \theint.
    \ifnum \thenumpofq<1
    \fillwithdottedlines{\@spaceperpart}
    \else
    \setcounter{iint}{0}
    {\loop
        \addtocounter{iint}{1}
        (\alph{iint})
        \fillwithdottedlines{\@spaceperpart}
        \ifnum \theiint<\thenumpofq
        \repeat
    }
    \fi
    \ifnum \theint<\numquestions
    \repeat
}
%</advanced&answerbook>
%    \end{macrocode}

%\subsubsection{Very advanced answer book}
%    \begin{macrocode}
%<*veryadvanced&answerbook>
\renewcommand{\answerbookpages}{
    \newcounter{arraycard}
    \def\arrayLength##1{%
        \setcounter{arraycard}{0}%
        \foreach \x in ##1{%
            \stepcounter{arraycard}%
        }%
        %
    }
    
    \newcommand{\setcountertoarraylength}[2]{
        \arrayLength{##2}
        \setcounter{##1}{\thearraycard}
    }
    
    \newcounter{max}
    \setcountertoarraylength{max}{\@questionorpartnumber}
    
    \xdef\qs{{\@questionorpartnumber}}
    \xdef\lines{{\@lines}}
    \xdef\ip{{\@partlevel}}
    \newcounter{numlines}
    \newcounter{quesnum}
    \newcounter{iP}
    
    \newcounter{int}
    \setcounter{int}{1}		
    
    \loop
    \pgfmathsetcounter{quesnum}{\qs[\theint]}
    \pgfmathsetcounter{numlines}{\lines[\theint]}
    \pgfmathsetcounter{iP}{\ip[\theint]}
    \ifnum \theiP=1
    (\alph{quesnum})
    \else
    \ifnum \thequesnum=1 \else \fillwithdottedlines{\stretch{1}} 
    \newpage \fi
    \thequesnum.
    \fi
    \ifnum \thenumlines=0 \else 
    \fillwithdottedlines{\thenumlines\linefillheight} \fi
    \addtocounter{int}{1}
    \ifnum \theint<\themax
    \repeat
    \fillwithdottedlines{\stretch{1}}
    \newpage
}

%</veryadvanced&answerbook>
%    \end{macrocode}
% \Finale
\endinput